% Glossary entries
\newglossaryentry{api} {
    name=\glslink{api}{API},
    text=Application Program Interface,
    sort=api,
    description={in informatica con il termine \emph{Application Programming Interface API} (ing. interfaccia di programmazione di un'applicazione) si indica ogni insieme di procedure disponibili al programmatore, di solito raggruppate a formare un set di strumenti specifici per l'espletamento di un determinato compito all'interno di un certo programma. La finalità è ottenere un'astrazione, di solito tra l'hardware e il programmatore o tra software a basso e quello ad alto livello semplificando così il lavoro di programmazione}
}

\newglossaryentry{dam}{
    name=\glslink{dam}{DAM},
    text=DAM,
    sort=dam,
    description={in informatica con il termine \emph{Digital Asset Management} (ing. gestione delle risorse digitali) si indica la gestione di contenuti multimediali (testi, immagini, video, audio) in formato digitale. Il software \emph{DAM} è un sistema di archiviazione e recupero di contenuti digitali, che permette di organizzare, catalogare, ricercare, condividere e distribuire i contenuti digitali in modo efficiente}
}

\newglossaryentry{pim}{
    name=\glslink{pim}{PIM},
    text=PIM,
    sort=pim,
    description={in informatica con il termine \emph{Product Information Management} (ing. gestione delle informazioni di prodotto) si indica la gestione delle informazioni relative ai prodotti di un'azienda. Il software \emph{PIM} è un sistema di archiviazione e recupero di informazioni, che permette di organizzare, catalogare, ricercare, condividere e distribuire le informazioni relative ai prodotti in modo efficiente}
}

\newglossaryentry{Agile}{
    name=\glslink{Agile}{Agile},
    text=Agile,
    sort=agile,
    description={In ingegneria del software, con il termine \emph{Agile} si indica un approccio metodologico che enfatizza la collaborazione e l’adattabilità nei processi di lavoro. L’\emph{Agile}
            si basa su un insieme di principi descritti nell’\emph{Agile Manifesto}, che promuove
            valori come la risposta ai cambiamenti dei requisiti o il coinvolgimento attivo del
            cliente}
}


\newglossaryentry{framework}{
    name=\glslink{framework}{framework},
    text=framework,
    sort=framework,
    description={In informatica, con il termine \emph{framework} si indica un'architettura logica di supporto su cui un software può essere progettato e realizzato, spesso facilitandone lo sviluppo da parte del programmatore. Un \emph{framework} può includere supporto per le librerie, i compilatori, gli strumenti di debug, i tool per la generazione di codice, e una serie di componenti che facilitano lo sviluppo di applicazioni}
}


\newglossaryentry{scrum}{
    name=\glslink{scrum}{scrum},
    text=scrum,
    sort=scrum,
    description={In ingegneria del software, Scrum è un \emph{framework agile} per la gestione del ciclo di sviluppo del software, iterativo ed incrementale, concepito per gestire progetti e prodotti software o applicazioni di sviluppo}
}

\newglossaryentry{Sprint}{
    name=\glslink{Sprint}{Sprint},
    text=Sprint,
    sort=Sprint,
    description={ In ingegneria del software, con il termine \emph{Sprint} si indica un'iterazione di sviluppo, di durata fissa, che produce un incremento di prodotto }
}

\newglossaryentry{frontend}{
    name=\glslink{frontend}{frontend},
    text=frontend,
    sort=frontend,
    description={In informatica, con il termine \emph{frontend} si indica la parte di un sistema software che gestisce l'interazione con l'utente o con sistemi esterni che producono dati di ingresso}
}


\newglossaryentry{backend}{
    name=\glslink{backend}{backend},
    text=backend,
    sort=backend,
    description={In informatica, con il termine \emph{backend} si indica la parte di un sistema software che elabora i dati di ingresso e genera i dati di uscita
        }
}

\newglossaryentry{repository}{
    name=\glslink{repository}{repository},
    text=repository,
    sort=repository,
    description={In informatica con il termine \emph{repository} si identifica
            un ambiente di archiviazione centralizzato dedicato alla gestione del codice
            sorgente e al suo versionamento}
}


\newglossaryentry{build}{
    name=\glslink{build}{build},
    text=build,
    sort=build,
    description={In informatica con il termine \emph{build} si indica il
            processo di compilazione di un programma, ovvero la traduzione del codice
            sorgente in codice eseguibile}
}

\newglossaryentry{deploy}{
    name=\glslink{deploy}{deploy},
    text=deploy,
    sort=deploy,
    description={In informatica con il termine \emph{deploy} si indica il
            processo di distribuzione di un programma, ovvero la messa in esercizio di
            un'applicazione in un ambiente operativo
        }
}

\newglossaryentry{UML}{
    name=\glslink{UML}{UML},
    text=UML,
    sort=UML,
    description={In ingegneria del software \emph{Unified Modelling Language
                UML} (ing. linguaggio di modellazione unififcato) è un linguaggio di
            modellazione al fine di fornire un insieme di simboli grafici, notazioni e
            convenzioni utili a rappresentare visivamente in modo chiaro il
            comportamento di un sistema software}
}

\newglossaryentry{open-source}{
    name=\glslink{open-source}{open-source},
    text=open-source,
    sort=open-source,
    description={In informatica con il termine \emph{open-source} si indica un
            software di cui gli autori (più precisamente i detentori dei diritti) ne
            rendono pubblico il codice sorgente, permettendo a chiunque di leggerlo ed
            utilizzarlo seguendo la licenza specificata}
}

\newglossaryentry{serverless}{
    name=\glslink{serverless}{serverless},
    text=serverless,
    sort=serverless,
    description={In informatica con il termine \emph{serverless} si indica un
            modello di sviluppo e distribuzione di applicazioni in cui il fornitore
            del servizio (ad esempio \emph{AWS}) si fa carico dell'esecuzione del
            codice, scalando automaticamente le risorse necessarie per eseguirlo}
}

\newglossaryentry{DynamoDB}{
    name=\glslink{DynamoDB}{DynamoDB},
    text=DynamoDB,
    sort=DynamoDB,
    description={In informatica con il termine \emph{DynamoDB} si indica un
            database \emph{NoSQL} completamente gestito, che supporta strutture di
            dati di tipo documento e di tipo chiave-valore, ed è offerto da
            \emph{AWS} come servizio \emph{serverless}}
}

\newglossaryentry{dashboard}{
    name=\glslink{dashboard}{dashboard},
    text=dashboard,
    sort=dashboard,
    description={In informatica con il termine \emph{dashboard} si indica una
            schermata che permette di visualizzare in modo sintetico e intuitivo
            informazioni e dati di interesse, in modo da facilitarne la lettura e
            l'interpretazione}
}

\newglossaryentry{CloudWatch}{
    name=\glslink{CloudWatch}{CloudWatch},
    text=CloudWatch,
    sort=CloudWatch,
    description={In informatica con il termine \emph{CloudWatch} si indica un
            servizio di monitoraggio e gestione delle risorse \emph{AWS} e delle
            applicazioni eseguite su \emph{AWS}, che permette di raccogliere e
            monitorare dati di interesse, visualizzandoli in modo sintetico e
            intuitivo}
}

\newglossaryentry{AWS}{
    name=\glslink{AWS}{AWS},
    text=AWS,
    sort=AWS,
    description={In informatica con il termine \emph{AWS} si indica
            \emph{Amazon Web Services}, un insieme di servizi di cloud computing
            offerti da \emph{Amazon} che compongono la piattaforma on demand più
            utilizzata al mondo}
}

\newglossaryentry{upscaling}{
    name=\glslink{upscaling}{upscaling},
    text=upscaling,
    sort=upscaling,
    description={In informatica con il termine \emph{upscaling} si indica un
            processo di conversione di un'immagine a bassa risoluzione in un'immagine
            ad alta risoluzione}
}

\newglossaryentry{software}{
    name=\glslink{software}{software},
    text=software,
    sort=software,
    description={In informatica con il termine \emph{software} si indica
            l'insieme dei programmi, codici e istruzioni digitali che compongono un
            sistema informatico.}
}

\newglossaryentry{metadati}{
    name=\glslink{metadati}{metadati},
    text=metadati,
    sort=metadati,
    description={In informatica con il termine \emph{metadati} si indica un
            insieme di informazioni che descrivono altre informazioni, come ad esempio
            nel caso delle immagini, il
            nome, il formato, la dimensione, la data di creazione, ecc.}
}

\newglossaryentry{EXIF}{
    name=\glslink{EXIF}{EXIF},
    text=EXIF,
    sort=EXIF,
    description={In informatica con il termine \emph{EXIF} si indica un
            formato di file utilizzato per memorizzare informazioni relative a una
            fotografia o a un'immagine digitale.}
}

\newglossaryentry{CLI}{
    name=\glslink{CLI}{CLI},
    text=CLI,
    sort=CLI,
    description={In informatica con il termine \emph{CLI} si indica un
            interfaccia utente che permette di interagire con un sistema informatico
            attraverso la scrittura di comandi testuali direttamente da terminale}
}

\newglossaryentry{AWS Cloudformation}{
    name=\glslink{AWS Cloudformation}{AWS Cloudformation},
    text=AWS Cloudformation,
    sort=AWS Cloudformation,
    description={Con il termine \emph{AWS Cloudformation} si
            indica un servizio di \emph{AWS} che permette di creare e gestire
            risorse \emph{AWS} attraverso la scrittura di codice \emph{JSON} o
            \emph{YAML}}
}

\newglossaryentry{docker}{
    name=\glslink{docker}{docker},
    text=Docker,
    sort=Docker,
    description={Con il termine \emph{docker} si indica un
            progetto open-source che automatizza il deployment di applicazioni
            all'interno di contenitori software isolati, fornendo un'astrazione aggiuntiva
            grazie alla virtualizzazione}
}

\newglossaryentry{container}{
    name=\glslink{container}{container},
    text=container,
    sort=container,
    description={In informatica con il termine \emph{container} si indica
            un'unità standard di software che raggruppa il codice e tutte le sue
            dipendenze, in modo da poter essere eseguito in modo affidabile e
            consistente su qualsiasi macchina}
}

\newglossaryentry{EFS}{
    name=\glslink{EFS}{EFS},
    text=EFS,
    sort=EFS,
    description={Con il termine \emph{EFS} si indica un servizio di
            \emph{AWS} che permette di creare e gestire un file system condiviso
            tra più istanze di computazione}
}

\newglossaryentry{DAO}{
    name=\glslink{DAO}{DAO},
    text=DAO,
    sort=DAO,
    description={In informatica con il termine \emph{DAO} (Data Access Object) si indica un
            pattern architetturale che permette di separare le logiche di business
            dalle logiche di accesso ai dati, in modo da poter essere utilizzate
            indipendentemente}
}

\newglossaryentry{mimetype}{
    name=\glslink{mimetype}{mimetype},
    text=mimetype,
    sort=mimetype,
    description={Con il termine \emph{mimetype} si indica
            un identificatore standard che permette di identificare il formato di un
            file in base al suo contenuto}
}