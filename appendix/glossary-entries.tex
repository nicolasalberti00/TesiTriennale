% Acronyms
\newacronym[description={\glslink{apig}{Application Program Interface}}]
{api}{API}{Application Program Interface}

\newacronym[description={\glslink{umlg}{Unified Modeling Language}}]
{uml}{UML}{Unified Modeling Language}

% Glossary entries
\newglossaryentry{apig} {
    name=\glslink{api}{API},
    text=Application Program Interface,
    sort=api,
    description={in informatica con il termine \emph{Application Programming Interface API} (ing. interfaccia di programmazione di un'applicazione) si indica ogni insieme di procedure disponibili al programmatore, di solito raggruppate a formare un set di strumenti specifici per l'espletamento di un determinato compito all'interno di un certo programma. La finalità è ottenere un'astrazione, di solito tra l'hardware e il programmatore o tra software a basso e quello ad alto livello semplificando così il lavoro di programmazione}
}

\newglossaryentry{umlg} {
    name=\glslink{uml}{UML},
    text=UML,
    sort=uml,
    description={in ingegneria del software \emph{UML, Unified Modeling Language} (ing. linguaggio di modellazione unificato) è un linguaggio di modellazione e specifica basato sul paradigma object-oriented. L'\emph{UML} svolge un'importantissima funzione di ``lingua franca'' nella comunità della progettazione e programmazione a oggetti. Gran parte della letteratura di settore usa tale linguaggio per descrivere soluzioni analitiche e progettuali in modo sintetico e comprensibile a un vasto pubblico}
}

\newglossaryentry{dam}{
    name=\glslink{dam}{DAM},
    text=DAM,
    sort=dam,
    description={in informatica con il termine \emph{Digital Asset Management} (ing. gestione delle risorse digitali) si indica la gestione di contenuti multimediali (testi, immagini, video, audio) in formato digitale. Il software \emph{DAM} è un sistema di archiviazione e recupero di contenuti digitali, che permette di organizzare, catalogare, ricercare, condividere e distribuire i contenuti digitali in modo efficiente}
}

\newglossaryentry{pim}{
    name=\glslink{pim}{PIM},
    text=PIM,
    sort=pim,
    description={in informatica con il termine \emph{Product Information Management} (ing. gestione delle informazioni di prodotto) si indica la gestione delle informazioni relative ai prodotti di un'azienda. Il software \emph{PIM} è un sistema di archiviazione e recupero di informazioni, che permette di organizzare, catalogare, ricercare, condividere e distribuire le informazioni relative ai prodotti in modo efficiente}
}

\newglossaryentry{Agile}{
    name=\glslink{Agile}{Agile},
    text=Agile,
    sort=agile,
    description={In ingegneria del software, con il termine \emph{Agile} si indica un approccio metodologico che enfatizza la collaborazione e l’adattabilità nei processi di lavoro. L’\emph{Agile}
            si basa su un insieme di principi descritti nell’\emph{Agile Manifesto}, che promuove
            valori come la risposta ai cambiamenti dei requisiti o il coinvolgimento attivo del
            cliente. }
}


\newglossaryentry{framework}{
    name=\glslink{framework}{framework},
    text=framework,
    sort=framework,
    description={In informatica, con il termine \emph{framework} si indica un'architettura logica di supporto su cui un software può essere progettato e realizzato, spesso facilitandone lo sviluppo da parte del programmatore. Un \emph{framework} può includere supporto per le librerie, i compilatori, gli strumenti di debug, i tool per la generazione di codice, e una serie di componenti che facilitano lo sviluppo di applicazioni}
}


\newglossaryentry{scrum}{
    name=\glslink{scrum}{scrum},
    text=scrum,
    sort=scrum,
    description={In ingegneria del software, Scrum è un \emph{framework agile} per la gestione del ciclo di sviluppo del software, iterativo ed incrementale, concepito per gestire progetti e prodotti software o applicazioni di sviluppo. }
}

\newglossaryentry{Sprint}{
    name=\glslink{Sprint}{Sprint},
    text=Sprint,
    sort=Sprint,
    description={ In ingegneria del software, con il termine \emph{Sprint} si indica un'iterazione di sviluppo, di durata fissa, che produce un incremento di prodotto. }
}

\newglossaryentry{frontend}{
    name=\glslink{frontend}{frontend},
    text=frontend,
    sort=frontend,
    description={In informatica, con il termine \emph{frontend} si indica la parte di un sistema software che gestisce l'interazione con l'utente o con sistemi esterni che producono dati di ingresso. }
}


\newglossaryentry{backend}{
    name=\glslink{backend}{backend},
    text=backend,
    sort=backend,
    description={In informatica, con il termine \emph{backend} si indica la parte di un sistema software che elabora i dati di ingresso e genera i dati di uscita.
        }
}

\newglossaryentry{repository}{
    name=\glslink{repository}{repository},
    text=repository,
    sort=repository,
    description={In informatica con il termine \emph{repository} si identifica
            un ambiente di archiviazione centralizzato dedicato alla gestione del codice
            sorgente e al suo versionamento.}
}


\newglossaryentry{build}{
    name=\glslink{build}{build},
    text=build,
    sort=build,
    description={In informatica con il termine \emph{build} si indica il
            processo di compilazione di un programma, ovvero la traduzione del codice
            sorgente in codice eseguibile.}
}

\newglossaryentry{deploy}{
    name=\glslink{deploy}{deploy},
    text=deploy,
    sort=deploy,
    description={In informatica con il termine \emph{deploy} si indica il
            processo di distribuzione di un programma, ovvero la messa in esercizio di
            un'applicazione in un ambiente operativo.
        }
}

\newglossaryentry{UML}{
    name=\glslink{UML}{UML},
    text=UML,
    sort=UML,
    description={In ingegneria del software \emph{Unified Modelling Language
                UML} (ing. linguaggio di modellazione unififcato) è un linguaggio di
            modellazione al fine di fornire un insieme di simboli grafici, notazioni e
            convenzioni utili a rappresentare visivamente in modo chiaro il
            comportamento di un sistema software.}
}
