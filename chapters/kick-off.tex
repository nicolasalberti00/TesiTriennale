\chapter{Descrizione dello stage}
\label{cap:descrizione-stage}

\intro{Il seguente capitolo vuole introdurre il progetto di stage, descrivendo
    gli obiettivi prefissati e i possibili rischi collegati allo sviluppo.}\\

\section{Introduzione al progetto}
Uno dei micro servizi offerti dalla piattaforma THRON è quello riguradante la
conversione delle immagini da un formato ad un altro, scalandole in base alla
risoluzione desiderata dal cliente. La conversione riguarda molteplici formati,
e dimensioni, e deve essere effettuata in maniera rapida ed efficiente.
THRON dispone già di un servizio di conversione immagini, ma si vuole
sostituirlo con uno nuovo, più efficiente e mantenibile. \\
Il progetto di stage è consistito nella realizzazione di un nuovo servizio di
conversione di immagini in \emph{Go}, utilizzando diverse librerie ed
applicazioni \emph{\glsfirstoccur\gls{open-source}}, integrandole in un unico servizio
capace di gestire diversi tipi di immagine.\\

\section{Obiettivi dello stage}
\label{sec:obiettivi}
In questa sezione vengono elencati gli obiettivi da raggiungere durante lo
stage, suddivisi in tre categorie: obbligatori, desiderabili e opzionali.\\
Per fare riferimento ai requisiti verranno utilizzate le seguenti notazioni:
\begin{itemize}
    \item \textbf{OB}: per i requisiti obbligatori, vincolanti in quanto obiettivo
          primario richiesto;
    \item \textbf{DE}: per i requisiti desiderabili, non strettamente necessari
          ma dal riconoscibile valore aggiunto;
    \item \textbf{OP}: per i requisiti opzionali, rappresentanti un valore
          aggiunto ma non necessariamente competitivo.
\end{itemize}
Le sigle sopra descritte saranno seguite da un codice identificativo univoco,
formato da una coppia di numeri, al fine di garantirne l'identificazione.\\
\subsection*{\emph{Obbligatori}}
\begin{itemize}
    \item \textbf{OB1}: realizzazione di un servizio di conversione immagini
          in \emph{Go} per supportare la migrazione aziendale in ambito
          \emph{\glsfirstoccur\gls{serverless}};
    \item \textbf{OB2:} effettuare la conversione seguendo un profilo di
          conversione recuperato da una tabella \emph{\glsfirstoccur\gls{DynamoDB}}
    \item \textbf{OB3:} scrivere su una tabella \emph{DynamoDB} il risultato
          della conversione, i dati specifici della conversione e le eventuali cause
          di errore;
\end{itemize}
\subsection*{\emph{Desiderabili}}
\begin{itemize}
    \item \textbf{DE1:} individuare il numero di immagini totali convertite, da
          rappresentare per cliente, per formato in input e per lo stato della
          conversione;
    \item \textbf{DE2:} individuare il tempo di esecuzione minimo, medio e
          massimo di una conversione, da rappresentare per cliente e per tipo di
          formato in input;
\end{itemize}
\subsection*{\emph{Opzionali}}
\begin{itemize}
    \item \textbf{OP1:} creare una \emph{\glsfirstoccur\gls{dashboard}} su
          \emph{\glsfirstoccur\gls{CloudWatch}} che permetta di monitorare il numero
          di errori nel tempo e gli obiettivi \textbf{DE1} e \textbf{DE2};
\end{itemize}
\section{Analisi preventiva dei rischi}

Durante la fase di analisi iniziale sono stati individuati alcuni possibili rischi a cui si potrà andare incontro.
Si è quindi proceduto a elaborare delle possibili soluzioni per far fronte a tali rischi.\\

\begin{risk}{\emph{Stack} tecnologico:}
    \riskdescription{le tecnologie utilizzate per lo sviluppo del progetto sono
        a me nuove. Ciò può portare ad un utilizzo non ottimale delle stesse, non
        rispettando le \emph{best practices} di riferimento, anche su tecnologie
        individuate in autonomia}
    \risksolution{è stato previsto un periodo di formazione iniziale, durante il quale ho potuto studiare in maniera autonoma e assieme al tutor aziendale le tecnologie da utilizzare e sperimentarle in un progetto di prova, su cui poi si sarebbe basato il progetto finale}
    \label{risk:stack-tecnologico}
\end{risk}

\begin{risk}{Fattibilità dei requisiti:}
    \riskdescription{alcuni dei requisiti risultano complessi da implementare
        e non è chiaro se sia possibile implementarli in maniera completa entro la fine dello
        stage}
    \risksolution{è stato deciso con il tutor aziendale di effettuare giornalmente una verifica della situazione, al fine di valutare l'andamento del progetto e decidere quali soluzioni adottare di fronte alle criticità riscontrate}
    \label{risk:fattibilità}
\end{risk}

\begin{risk}{Ritardi nello sviluppo:}
    \riskdescription{a causa di attività esterne al progetto (come ad esempio le attività di preparazione dell'infrastruttura), potrebbero verificarsi ritardi nello sviluppo del progetto}
    \risksolution{è stato deciso con il tutor aziendale di effettuare una
        analisi settimanale della situazione ed in caso di criticità è stato
        previsto un dialogo con il \emph{Software Architect} del team di
        \emph{backend}, al fine di superare le criticità riscontrate}
    \label{risk:ritardi-sviluppo}
\end{risk}
