\chapter{Codifica}
\label{cap:codifica}

\intro{In questo capitolo viene descritto il processo di codifica del progetto di stage.}

\section{Codifica back-end}

In questo capitolo verrà descritto il processo di codifica \emph{back-end} del
progetto di stage. Verranno elencate tutte le funzionalità e la struttura del
codice implementato.

\subsection{Config}

All'interno di questa cartella è presente il file di configurazione utilizzato
dal servizio. Di seguito la descrizione dello stesso.

\subsubsection{\emph{config.go}}

Questo file contiene la configurazione necessaria al funzionamento del servizio.
Esso contiene al suo interno le variabili d'ambiente che verranno utilizzate
dalla funzione \emph{Lambda} per recuperare i nomi delle tabelle \emph{DynamoDB}
e i nomi dei \emph{bucket S3} dove verranno recuperate e salvate le immagini.

\subsection{Clients}

All'interno di questa cartella è presente il file che definisce le funzioni per
utilizzare i \emph{client} di \emph{DynamoDB} e \emph{S3}.

\subsubsection{\emph{clients}}

Questo file contiene le funzioni per utilizzare i \emph{client} di
\emph{DynamoDB} e \emph{S3}. Si dividono in funzioni \emph{remote} e
\emph{local}, più nello specifico:
\begin{itemize}
    \item \textbf{remote}: funzioni che utilizzano i \emph{client} di
          \emph{DynamoDB} e \emph{S3} per recuperare e salvare i dati nell'ambiente
          \emph{sandbox}. Queste funzioni utilizzano la configurazione di
          \emph{default} per accedere ai servizi \emph{AWS};
    \item \textbf{local}: funzioni che utilizzano i \emph{client} di
          \emph{DynamoDB} e \emph{S3} per recuperare e salvare i dati nell'ambiente di
          \emph{test} in locale. Queste funzioni utilizzano una configurazione vuota,
          senza credenziali, che punta ad un \emph{endpoint} locale.
\end{itemize}

\subsection{Cmd}

All'interno di questa cartella è presente il file che definisce le funzioni per
avviare la \emph{Lambda}.

\subsubsection{\emph{main}}

Questo file contiene la struttura che definisce un evento di conversione, nello
specifico è così rappresentata:
\begin{lstlisting}[language=go]
    type ImageConversionEvent struct {
        Filename string `json:"filename"`
	ClientID string `json:"clientID"`
    }
\end{lstlisting}

\captionof{lstlisting}{Struttura di un evento di conversione}

Nello specifico vengono definiti:
\begin{itemize}
    \item \textbf{Filename}: nome del file da convertire;
    \item \textbf{ClientID}: identificativo del cliente che ha richiesto la
          conversione.
\end{itemize}
Entrambi vengono recuperati da un file \emph{JSON} che viene passato alla
funzione \emph{Lambda}: questo file nel progetto di stage viene inserito
manualmente, in un ambiente di produzione verrà fornito da un servizio
esterno.\\
Nella funzione \emph{main} viene recuperata la configurazione desiderata e
vengono istanziati i \glsfirstoccur\gls{DAO}, gli oggetti che permettono di
interagire con i servizi \emph{S3} e \emph{DynamoDB}. Successivamente viene
inizializzato il servizio che effettua la conversione delle immagini e viene
passato alla funzione \emph{Lambda}.

\subsection{DAL}

All'interno di questa cartella sono presenti i file che gestiscono gli accessi
agli oggetti presenti nei \emph{bucket S3} e nei \emph{database DynamoDB}.

\subsubsection{\emph{client configuration DAO}}

Questo file contiene la struttura che definisce un \emph{DAO} per il recupero di
una configurazione di un cliente. Con il \emph{DAO} si aggiunge un livello di
astrazione tra due parti di una applicazione che devono lavorare a stretto
contatto, restando indipendenti l'una dall'altra.\\
Questo \emph{design pattern} permette inoltre di separare la logica di accesso
ai servizi con cui si interfaccia, così nel caso in cui si volesse cambiare il
servizio sottostante si dovrà solamente cambiare il \emph{DAO} e non l'intera
applicazione. La struttura di una configurazione è la seguente:
\begin{lstlisting}[language=go]
type ClientConfiguration struct {
        ConfigurationName string `dynamodbav:"ConfigurationName"`
        Width   string `dynamodbav:"Width"`
        Height  string `dynamodbav:"Height"`
        Quality int    `dynamodbav:"Quality"`
    }
\end{lstlisting}
L'attributo \emph{dynamodbav} permette al \emph{client} di \emph{DynamoDB} di
mappare i campi della struttura con i campi della tabella.\\

\subsubsection{\emph{conversions bucket DAO}}

Questo file contiene la struttura che definisce un \emph{DAO} per effettuare il
\emph{download} e l'\emph{upload} di un file da un \emph{bucket S3}. La funzione
di \emph{download} prende in input tre parametri:
\begin{itemize}
    \item \textbf{clientID:} identificativo del cliente che ha richiesto la
          conversione;
    \item \textbf{jobID:} identificativo della conversione;
    \item \textbf{filename:} nome del file da scaricare.
\end{itemize}

Viene individuato il file all'interno del \emph{bucket S3} delle immagini da convertire, che si trova nella
cartella specificata dal \emph{clientID}. Successivamente viene effettuata una
copia del file in un percorso locale e viene restituito il percorso del file
appena salvato.\\

La funzione di \emph{upload} prende in input due parametri:
\begin{itemize}
    \item \textbf{clientID:} identificativo del cliente che ha richiesto la
          conversione;
    \item \textbf{filename:} nome del file da caricare.
\end{itemize}
Grazie a questi due parametri viene effettuato il caricamento del file nel
\emph{bucket} destinato alle immagini convertite.

\subsubsection{\emph{image conversion jobs DAO}}

Questo file contiene la struttura che definisce un \emph{DAO} per inserire le
specifiche di una conversione oppure per aggiornarle. La struttura di una
conversione è la seguente:
\begin{lstlisting}[language=go]
    type ImageConversionJob struct {
        ClientID       string `dynamodbav:"ClientID"`
        CreationDate   int64  `dynamodbav:"CreationDate"`
        ContentType    string `dynamodbav:"ContentType"`
        JobID          string `dynamodbav:"JobID"`
        EndDate        int64  `dynamodbav:"EndDate"`
        Status         Status `dynamodbav:"Status"`
        SourcePath     string `dynamodbav:"SourcePath"`
        TotalTime      int64  `dynamodbav:"TotalTime"`
        ConversionTime int64  `dynamodbav:"ConversionTime"`
        ErrorCause     string `dynamodbav:"ErrorCause"`
    }
\end{lstlisting}

Vengono salvate tutte le informazioni riguardanti la conversione, nello
specifico vengono salvati:
\begin{itemize}
    \item \textbf{ClientID:} identificativo del cliente che ha richiesto la
          conversione;
    \item \textbf{CreationDate:} data di inizio della conversione;
    \item \textbf{ContentType:} formato del file da convertire;
    \item \textbf{JobID:} identificativo della conversione;
    \item \textbf{EndDate:} data di fine della conversione;
    \item \textbf{Status:} stato della conversione. Può assumere tre valori: \emph{RUNNING}, \emph{COMPLETE} e \emph{ERROR};
    \item \textbf{SourcePath:} percorso del file da convertire;
    \item \textbf{TotalTime:} tempo totale della conversione, compreso il tempo
          di \emph{download} e \emph{upload};
    \item \textbf{ConversionTime:} tempo di conversione;
    \item \textbf{ErrorCause:} causa dell'errore, se presente.
\end{itemize}

\subsection{Service}

In questa cartella viene definito il file principale del servizio, che contiene
il flusso di conversione delle immagini.

\subsubsection{\emph{image conversion service}}

In questo file viene definito tutto il flusso di conversione dell'immagine che
verrà poi gestito dalla \emph{Lambda}.\\
Sono state definite delle strutture che contengono tutti i formati gestiti dal
servizio, suddivisi per libreria che gestisce la conversione. Questi vengono
utilizzati per effettuare i controlli utilizzati come guardia per scegliere la
conversione adatta al formato richiesto. Per accelerare il processo di
conversione, seguendo il principio \emph{fail fast}, se il formato in ingresso
non è presente nelle strutture viene interrotto il flusso e viene restituito un
errore.\\
Se il formato è accettato, viene verificato se è in atto un tentativo di
\emph{upscaling} dell'immagine, confrontando le dimensioni dell'immagine con
quelle richieste dalla configurazione: se sta per essere effettuato, viene
saltata la conversione per quel formato e viene restituito un messaggio di
\emph{warning} per avvisare che quella conversione non verrà eseguita.\\
Viene effettuata la conversione utilizzando il metodo adatto al formato in
ingresso ed infine viene caricata l'immagine nel \emph{bucket} di output ed
aggiornata la tabella contenente i \emph{job} delle conversioni.


