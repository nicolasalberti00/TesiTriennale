\chapter{Introduzione}
\label{cap:introduzione}

\intro{Il seguente capitolo vuole introdurre brevemente l'azienda e il progetto di
      stage.}
%
%\noindent Esempio di utilizzo di un termine nel glossario \\
%\gls{api}. \\

%\noindent Esempio di citazione in linea \\
%\cite{site:agile-manifesto}. \\

%\noindent Esempio di citazione nel pie' di pagina \\
%citazione\footcite{womak:lean-thinking} \\

\section{L'azienda}

THRON S.p.A. è un'azienda italiana con sede a Piazzola sul Brenta i cui prodotti
principali sono la \textit{THRON DAM Platform} e il \textit{THRON PIM (Product
      Information Management)}. Il primo prodotto è una soluzione che permette di
gestire, organizzare e distribuire i contenuti digitali di una azienda in modo
efficace, mentre il secondo è una soluzione che permette di creare, raggruppare,
organizzare e distribuire le informazioni relative ai prodotti e ai contenuti
multimediali di un'azienda.\\
L'obiettivo è garantire una gestione centralizzata dei contenuti e semplificare
la gestione degli \textit{asset} digitali, permettendone l'adattamento e la
distribuzione in modo semplice ed efficiente.\\
L'area \textit{Research and Developement} (Ricerca e Sviluppo) è suddivisa in due team: il team Prodotti, responsabile
della gestione del \glsfirstoccur\gls{pim}, e il team Contenuti, responsabile
delle tematiche legate alla gestione del \glsfirstoccur\gls{dam}.\\
\section{Metodologie di sviluppo}

In THRON vengono utilizzate metodologie di sviluppo \glsfirstoccur\gls{Agile}, in cui si pone
l'accento sulla collaborazione tra i membri del team, al fine di reagire in
maniera rapida ai cambiamenti e permettere la consegna di prodotti di qualità.\\

Il \glsfirstoccur\gls{framework} di riferimento è \glsfirstoccur\gls{scrum}, che prevede la suddivisione del
lavoro in iterazioni chiamate \glsfirstoccur\gls{Sprint}. Ogni sprint ha una durata di una
settimana (?), al cui termine viene organizzata una riunione denominata
\emph{"Retrospettiva"}, in cui si analizza il lavoro svolto durante lo sprint
e si valuta se è stato raggiunto l'obiettivo prefissato. Ogni
membro del team indica quali argomenti sono stati positivi e quali
negativi, al fine di migliorare le modalità di lavoro nello
\emph{Sprint} successivo.\\
Una importante pratica adottata è quella del \emph{"Daily
      meeting"}, una riunione in cui ogni membro del team espone il proprio
avanzamento, le difficoltà incontrate e le attività che intende svolgere. \\
Un'altra pratica adottata è quella del \emph{"Competence meeting"}, una riunione
in cui vengono riuniti i membri di THRON che condividono la stessa area di
competenza, come \glsfirstoccur\gls{frontend} o \glsfirstoccur\gls{backend}, al
fine di discutere e condividere le conoscenze e le esperienze maturate.\\

\section{Strumenti di sviluppo}
Durante l'esperienza in THRON è stato fondamentale utilizzare strumenti che
potessero guidare lo sviluppo in maniera efficace ed efficiente.\\
Di seguito vengono elencati gli strumenti utilizzati durante lo sviluppo del
progetto:
\begin{itemize}
      \item \textbf{Microsoft Teams}: per le comunicazioni sincrone ed asincrone
            con i membri del team;
      \item \textbf{Microsoft Office 365}: per la gestione della mail aziendale;
      \item \textbf{\glsfirstoccur\gls{AWS}}: per la gestione del \glsfirstoccur\gls{repository} del
            progetto, per la \glsfirstoccur\gls{build} e il \glsfirstoccur\gls{deploy}
            dello stesso;
      \item \textbf{Confluence}: per la gestione della documentazione del progetto;
      \item \textbf{StarUML}: per la modellazione dei diagrammi \glsfirstoccur\gls{UML}.
\end{itemize}

\section{Organizzazione del testo}

\begin{description}
      \item[{\hyperref[cap:descrizione-stage]{Il secondo capitolo}}] descrive
            il progetto svolto durante il periodo di stage, mostrando gli
            obiettivi imposti dall'azienda e i possibili rischi collegati.

      \item[{\hyperref[cap:analisi-requisiti]{Il terzo capitolo}}] descrive
            l'analisi dei requisiti del progetto di stage, mostrando come vengono
            modellati i casi d'uso e i requisiti tratti da essi.

      \item[{\hyperref[cap:struttura-progettazione]{Il quarto capitolo}}] descrive
            la struttura principale del progetto e le scelte progettuali effettuate
            durante lo sviluppo.

      \item[{\hyperref[cap:codifica]{Il quinto capitolo}}] descrive l'attività di
            codifica backend del progetto.

      \item[{\hyperref[cap:verifica-validazione]{Il sesto capitolo}}] descrive le
            attività di verifica e validazione sul progetto attraverso i test effettuati
            ed i conseguenti risultati ottenuti.

      \item[{\hyperref[cap:conclusioni]{Nel settimo capitolo}}] espone le
            conclusioni tratte dallo stage, descrivendo le conoscenze e le competenze
            acquisite seguite da una valutazione personale dell'esperienza svolta.
\end{description}

Riguardo la stesura del testo, relativamente al documentono state adottate le seguenti convenzioni tipografiche:
\begin{itemize}
      \item Gli acronimi, le abbreviazioni e i termini ambigui o di uso non comune menzionati vengono definiti nel glossario, situato alla fine del presente documento;
      \item Per la prima occorrenza dei termini riportati nel glossario viene utilizzata la seguente nomenclatura: \glsfirstoccur\emph{parola};
      \item I termini in lingua straniera o facenti parti del gergo tecnico sono evidenziati con il carattere \emph{corsivo}.
\end{itemize}
