\chapter{Introduzione}
\label{cap:introduzione}

\intro{Il seguente capitolo vuole introdurre brevemente l'azienda e il progetto di
    stage.}
%
%\noindent Esempio di utilizzo di un termine nel glossario \\
%\gls{api}. \\

%\noindent Esempio di citazione in linea \\
%\cite{site:agile-manifesto}. \\

%\noindent Esempio di citazione nel pie' di pagina \\
%citazione\footcite{womak:lean-thinking} \\

\section{L'azienda}

THRON S.p.A. è un'azienda italiana con sede a Piazzola sul Brenta i cui prodotti
principali sono la \textit{THRON DAM Platform} e il \textit{THRON PIM (Product
    Information Management)}. Il primo prodotto è una soluzione che permette di
gestire, organizzare e distribuire i contenuti digitali di una azienda in modo
efficace, mentre il secondo è una soluzione che permette di creare, raggruppare,
organizzare e distribuire le informazioni relative ai prodotti e ai contenuti
multimediali di un'azienda.\\
L'obiettivo è garantire una gestione centralizzata dei contenuti e semplificare
la gestione degli \textit{asset} digitali, permettendone l'adattamento e la
distribuzione in modo semplice ed efficiente.\\
L'area \textit{Research and Developement} (Ricerca e Sviluppo) è suddivisa in due team: il team Prodotti, responsabile
della gestione del \glsfirstoccur\gls{pim}, e il team Contenuti, responsabile
delle tematiche legate alla gestione del \glsfirstoccur\gls{dam}.\\
\section{Metodologie di sviluppo}

In THRON vengono utilizzate metodologie di sviluppo \glsfirstoccur\gls{Agile}, in cui si pone
l'accento sulla collaborazione tra i membri del team, al fine di reagire in
maniera rapida ai cambiamenti e permettere la consegna di prodotti di qualità.\\

Il \gls{framework} di riferimento è \gls{scrum}, che prevede la suddivisione del
lavoro in iterazioni chiamate \gls{Sprint}. Ogni sprint ha una durata di una
settimana (?), al termine dell

\section{Organizzazione del testo}

\begin{description}
    \item[{\hyperref[cap:processi-metodologie]{Il secondo capitolo}}] descrive ...

    \item[{\hyperref[cap:descrizione-stage]{Il terzo capitolo}}] approfondisce ...

    \item[{\hyperref[cap:analisi-requisiti]{Il quarto capitolo}}] approfondisce ...

    \item[{\hyperref[cap:progettazione-codifica]{Il quinto capitolo}}] approfondisce ...

    \item[{\hyperref[cap:verifica-validazione]{Il sesto capitolo}}] approfondisce ...

    \item[{\hyperref[cap:conclusioni]{Nel settimo capitolo}}] descrive ...
\end{description}

Riguardo la stesura del testo, relativamente al documento sono state adottate le seguenti convenzioni tipografiche:
\begin{itemize}
    \item gli acronimi, le abbreviazioni e i termini ambigui o di uso non comune menzionati vengono definiti nel glossario, situato alla fine del presente documento;
    \item per la prima occorrenza dei termini riportati nel glossario viene utilizzata la seguente nomenclatura: \emph{parola}\glsfirstoccur;
    \item i termini in lingua straniera o facenti parti del gergo tecnico sono evidenziati con il carattere \emph{corsivo}.
\end{itemize}
