\chapter{Analisi dei requisiti}
\label{cap:analisi-requisiti}

\intro{In questo capitolo viene presentata l'analisi dei requisiti effettuata
    durante lo stage, dove si mostrano le funzionalità e i requisiti individuati, al
    fine di fornire una visione più chiara del sistema.}\\

\section{Descrizione dell'applicazione}
Il progetto consiste nel creare un servizio per la
conversione di vari formati immagine. Il prodotto vuole essere una
sperimentazione interna per la sostituzione di un servizio interno che svolge
già lo stesso compito, utilizzando delle tecnologie \emph{cloud-native} e
\emph{serverless}, al fine di semplificare il flusso della conversione e di
fornire uno strumento più mantenibile e scalabile rispetto al precedente.\\


\section{Casi d'uso}

Per lo studio dei casi di utilizzo del prodotto sono stati creati dei diagrammi.
I diagrammi dei casi d'uso (in inglese \emph{Use Case Diagram}) sono diagrammi di tipo \gls{uml} dedicati
alla descrizione delle funzioni o servizi offerti da un sistema, così come sono
percepiti e utilizzati dagli attori che interagiscono col sistema stesso.
Essendo il progetto finalizzato alla creazione di un servizio che richiede
un'interazione minima, i diagrammi risultano semplici ed in numero ridotto.

\subsection{Descrizione del sistema}
Di seguito viene rappresentato un diagramma riassuntivo che mostra i casi d'uso
individuati e le relazioni tra essi.
\begin{figure}[!h]
    \centering

    \caption{Use Case - UC0: Scenario principale}
\end{figure}

\begin{usecase}{0}{Scenario principale}
    \usecaseactors{Sviluppatore applicativi}
    \usecasepre{Lo sviluppatore è entrato nel plug-in di simulazione all'interno dell'IDE}
    \usecasedesc{La finestra di simulazione mette a disposizione i comandi per configurare, registrare o eseguire un test}
    \usecasepost{Il sistema è pronto per permettere una nuova interazione}
    \label{uc:scenario-principale}
\end{usecase}

\section{Tracciamento dei requisiti}

Vengono di seguito presentati i requisiti individuati durante il progetto di
stage.\\
Sono stati individuati diversi tipi di requisiti e si è quindi fatto utilizzo di
un codice identificativo per distinguerli, rappresentato di seguito:
\begin{itemize}
    \item \textbf{Requisiti funzionali:} descrivono le funzioni che il sistema
          deve offrire. Delineano le azioni che il sistema deve eseguire, le
          risposte attese da determinati input e le dinamiche generali del sistema;
    \item \textbf{Requisiti qualitativi:} descrivono le caratteristiche che il
          sistema deve possedere, legati alla qualità e alle prestazioni del sistema;
    \item \textbf{Requisiti di vincolo:} descrivono i parametri che il sistema
          deve rispettare durante lo sviluppo e l'implementazione.
\end{itemize}
Viene inoltre fatta una classificazione dei requisiti in base alla loro
priorità.

\subsection{Notazione}
Ciascun requisito è identificato da un codice univoco, che aderisce alla
seguente notazione:
\begin{center}
    \textbf{R[Priorità][Tipo]-[Codice]}
\end{center}
dove:
\begin{itemize}
    \item \textbf{Priorità:} può assumere i seguenti valori:
          \begin{itemize}
              \item \textbf{O:} requisito obbligatorio;
              \item \textbf{D:} requisito desiderabile;
              \item \textbf{Z:} requisito opzionaleo.
          \end{itemize}
    \item \textbf{Tipo:} può assumere i seguenti valori:
          \begin{itemize}
              \item \textbf{F:} requisito funzionale;
              \item \textbf{Q:} requisito qualitativo;
              \item \textbf{V:} requisito di vincolo.
          \end{itemize}
    \item \textbf{Codice:} è un codice identificativo univoco del requisito.
\end{itemize}

Nelle tabelle \ref{tab:requisiti-funzionali}, \ref{tab:requisiti-qualitativi} e
\ref{tab:requisiti-vincolo} sono riassunti i requisiti e il loro tracciamento
con i casi d'uso delineati in fase di analisi.

\newpage

\subsection{Requisiti funzionali}
\begin{table}%
    \caption{Tabella del tracciamento dei requisti funzionali}
    \label{tab:requisiti-funzionali}
    \begin{tabularx}{\textwidth}{lXl}
        \hline\hline
        \textbf{Requisito} & \textbf{Descrizione}                         & \textbf{Use Case} \\
        \hline
        ROF-1              & Permettere una conversione da formato JPG al
        formato JPG        & UC1                                                              \\
        \hline
    \end{tabularx}
\end{table}%

\subsection{Requisiti qualitativi}
\begin{table}%
    \caption{Tabella del tracciamento dei requisiti qualitativi}
    \label{tab:requisiti-qualitativi}
    \begin{tabularx}{\textwidth}{lXl}
        \hline\hline
        \textbf{Requisito}                  & \textbf{Descrizione}                    & \textbf{Use Case} \\
        \hline
        RQO-1                               & Il progetto deve essere accompagnato da
        documentazione tecnica e funzionale & Interno                                                     \\
        \hline
    \end{tabularx}
\end{table}%

\subsection{Requisiti di vincolo}
\begin{table}%
    \caption{Tabella del tracciamento dei requisiti di vincolo}
    \label{tab:requisiti-vincolo}
    \begin{tabularx}{\textwidth}{lXl}
        \hline\hline
        \textbf{Requisito} & \textbf{Descrizione}                                                        & \textbf{Use Case} \\
        \hline
        RVO-1              & La libreria per l'esecuzione dei test automatici deve essere riutilizzabile & -                 \\
        \hline
    \end{tabularx}
\end{table}%
