\chapter{Conclusioni}
\label{cap:conclusioni}

Il progetto di stage svolto ha avuto come scopo la realizzazione di un servizio
per la conversione di immagini in linguaggio di programmazione \emph{Go}. Il
servizio doveva essere realizzato utilizzando tecnologie \emph{cloud-native} e
\emph{serverless}, quali \emph{AWS Lambda}, \emph{AWS S3} e \emph{AWS DynamoDB}.
Il servizio prevedeva l'utilizzo di diverse librerie e programmi specializzati,
al fine di mantenere il supporto a tutti i formati gestiti dalla piattaforma
THRON. Per la realizzazione del progetto è stato necessario seguire varie fasi
di sviluppo, quali l'analisi dei requisiti, in cui sono stati definiti i casi
d'uso e i requisiti previsti dal sistema. Successivamente, sono state analizzate
le tecnologie da utilizzare, con uno studio di esse, concentrandosi sul
linguaggio di programmazione \emph{Go} e la sua integrazione con i servizi
forniti da \emph{AWS}. Infine è stata progettata l'architettura del servizio,
permettendo la generazione delle risorse in maniera programmatica, ed
effettuando l'implementazione di tutte le funzionalità previste.\\

In merito agli obiettivi prefissati per lo stage (in sezione
\ref{sec:obiettivi}), sono stati raggiunti tutti gli obiettivi obbligatori. Per
quanto riguarda l'obiettivo opzionale non è stato raggiunto, in quanto è stato
dedicato più tempo alla fase di sviluppo della \emph{Proof of Concept} del
progetto e non è stato possibile implementare la \emph{dashboard} desiderata.
Gli obiettivi desiderabili invece sono stati raggiunti solo parzialmente, in
quanto sono state ed implementate le strutture per ospitare i dati necessari ad
ottenere tali informazioni, ma non è stato possibile implementare le
\emph{query} specifiche a causa della mancanza di tempo a disposizione.\\
Di seguito vengono riportati più in dettaglio gli obiettivi superati:
\begin{itemize}
      \item \textbf{OB1}: realizzazione di un servizio di conversione immagini
            in \emph{Go} per supportare la migrazione aziendale in ambito
            \emph{serverless};
      \item \textbf{OB2:} effettuare la conversione seguendo un profilo di
            conversione recuperato da una tabella \emph{DynamoDB}
      \item \textbf{OB3:} scrivere su una tabella \emph{DynamoDB} il risultato
            della conversione, i dati specifici della conversione e le eventuali cause
            di errore;
\end{itemize}

\section{Conoscenze acquisite}
Il progetto di stage mi ha permesso di acquisire nuove competenze e conoscenze
tecniche e personali, andando ad arricchire il mio bagaglio tecnico e culturale
e soddisfando le mie aspettative iniziali. È stato molto interessante poter conoscere le tecnologie \emph{THRON DAM} e
\emph{THRON PIM} di cui non ero a conoscenza, ed è stato molto stimolante
analizzare come funzionano servizi di gestione degli \emph{asset} digitali nel
loro lato più tecnico. Con questo ho potuto avere una conoscenza più ampia del
contesto operativo dell'azienda in qui ho svolto lo stage.\\
Tra le competenze tecniche spicca sicuramente l'apprendimento del linguaggio di
programmazione \emph{Go}, che non avevo mai utilizzato prima. Sono rimasto
sorpreso dalla semplicità e dalla potenza che offre il linguaggio agli
sviluppatori, permettendo uno sviluppo rapido ma allo stesso tempo
efficace. Ho sfruttato appieno le \emph{best practices} definite dalla
\emph{community} di sviluppatori \emph{Go} \cite{effective-go}, principalmente nell'attività di
gestione degli errori e nell'attività di \emph{logging}. Sono rimasto
soddisfatto della scelta di questo stage, in quanto uno dei motivi che mi ha
spinto a scegliere questa azienda e il suo progetto è stata proprio la
possibilità di utilizzare questo linguaggio di programmazione, che intendevo
scoprire e approfondire anche a livello personale.
Ho potuto conoscere ed approfondire le tecnologie \emph{AWS}, che non avevo mai
utilizzato prima di questo stage e che ho trovato estremamente interessanti e
complesse allo stesso tempo. È stato molto stimolante poter utilizzare delle
risorse di tipo \emph{cloud-native} e \emph{serverless} per la realizzazione di
un servizio simile, permettendomi di conoscere nuovi paradigmi di sviluppo e
nuove metodologie di lavoro.\\
La competenza personale più importante che ho acquisito è stata la capacità di
lavorare a stretto contatto con un team di sviluppo, partecipando attivamente a
tutte le loro attività, dalla riunione quotidiana fino alle riunioni di
\emph{retrospettiva}, in cui ho approfondito la mia capacità di analisi e
relazione dello sviluppo del lavoro svolto. Ho potuto inoltre sperimentare ed
approfondire la metodologia \emph{Scrum}, già utilizzata nel corso di
\emph{Ingegneria del Software}, ma applicata in maniera meno approfondita
rispetto a quanto visto in questo stage.

\section{Valutazione personale}

L'esperienza di stage è stata una parte fondamendale del mio percorso
universitario, sia a livello personale che professionale. In questo periodo ho
avuto la possibilità di affrontare in maniera più dettagliata l'ambito dello sviluppo
\emph{back-end}, che ho sempre trovato molto complesso da attuare. Grazie a
questo stage mi sono reso conto che le mie idee riguardanti questo sviluppo
erano fuorvianti e dettate forse dalla mia poca esperienza in questo ambito: ho
potuto constatare che lo sviluppo \emph{back-end} è un'attività molto stimolante
e che porta anche una certa gratificazione personale, dato che è stato bello in
questo caso vedere i progressi fatti man mano che il progetto prendeva forma.\\
Un altro aspetto che ho apprezzato molto è stato il lavoro in un team di
sviluppo ben consolidato, che segue regole ben definite al fine di garantire il
raggiungimento degli obiettivi prefissati in maniera efficace ed efficiente. La
presenza a stretto contatto del team mi ha permesso di integrare la loro
metodologia di lavoro anche nel mio progetto di stage, stimolandomi ad ottenere
risultati migliori e dimostrabili a tutti. Grazie alle interazioni con il gruppo
ho avuto la possiblità di comprendere meglio le mie passioni e di definire in
maniera più precisa il mio percorso di studi e di carriera professionale
futuri.\\
Il periodo di due mesi trascorso in azienda è stato molto stimolante e
piacevole, anche grazie all'azienda stessa che promuove un ambiente di lavoro
giovane, stimolante ma anche informale e rilassato. Le persone con cui sono
stato a contatto si sono sempre dimostrate disponibili e competenti, mi hanno
guidato ed aiutato durante questo periodo, permettendomi di raggiungere gli
obiettivi prefissati sia a livello tecnico sia a livello personale.\\
In conclusione, valuto l'esperienza di stage come estremamente positiva, gratificante e
necessaria all'interno di un percorso di studi come quello frequentato. È
un'attività assolutamente fondamentale, che permette a chi la svolge di mettersi
in gioco e di imparare competenze e conoscenze tecniche e non, richieste dal
mondo del lavoro. Sono estremamente soddisfatto del percorso seguito e dei
risultati ottenuti, che mi hanno permesso di arricchire il mio bagaglio tecnico
e culturale per affrontare al meglio le sfide che si presenteranno in futuro.