\chapter{Struttura e progettazione}
\label{cap:struttura-progettazione}

\intro{In questo capitolo viene presentata la struttura principale del progetto
    e le attività di progettazione del servizio. Vengono inoltre descritte le
    tecnologie utilizzate durante lo sviluppo e le scelte architetturali.}\\

\section{Tecnologie utilizzate}

Di seguito vengono elencate e descritte le tecnologie utilizzate durante lo sviluppo del
progetto di stage.

\subsection{Back-end}

\subsubsection{Go}

Go è un linguaggio di programmazione \emph{open source} sviluppato da
\emph{Google} nel 2007. È un linguaggio compilato, staticamente tipizzato e
fortemente tipizzato, con una sintassi simile a quella del linguaggio \emph{C}.
Il linguaggio è stato progettato per semplificare la programmazione di sistemi
distribuiti, con particolare attenzione alla concorrenza e alla
comunicazione in rete. \\
È un linguaggio che segue una filosofia di scrittura del codice molto semplice e
nel mio progetto ho seguito alcune delle linee guida proposte dalla comunità di
sviluppatori, come ad esempio la struttura del codice e la gestione degli errori
con il \emph{pattern} \emph{error handling}. \\
Per il mio progetto ho utilizzato la versione 1.21 di \emph{Go}, al fine di
utilizzare l'ultima versione attualmente disponibile. \cite{go}

\subsubsection{AWS CDK}

\emph{AWS Cloud Development Kit} (AWS CDK) è un \emph{framework} di sviluppo
\emph{open-source} che permette di definire le risorse \emph{AWS} utilizzando
uno o più linguaggi di programmazione. Questo \emph{framework} prevede il pieno
supporto al linguaggio di programmazione \emph{Go}. \\
Nel progetto è stato utilizzato per definire lo \emph{stack} contenente le risorse \emph{AWS}
necessarie al funzionamento del servizio, come la funzione \emph{Lambda}, le
tabelle \emph{DynamoDB} e i \emph{bucket S3}. \cite{go-aws-cdk}

\subsubsection{AWS SDK}

\emph{AWS Software Developement Kit} (AWS SDK) è un insieme di librerie che
permettono di interagire con i servizi \emph{AWS} utilizzando diversi linguaggi
di programmazione. L'\emph{SDK} prevede il pieno supporto al linguaggio di
programmazione \emph{Go}. \\
Nel progetto è stato utilizzato per interagire con le risorse \emph{AWS}
definite con \emph{AWS CDK}. \cite{go-aws-sdk}

\subsubsection{AWS Lambda}

\emph{AWS Lambda} è un servizio \emph{serverless} che permette di eseguire del
codice senza dover gestire l'infrastruttura sottostante. Il servizio permette di
eseguire del codice in risposta a determinati eventi, come ad esempio una
richiesta \emph{HTTP} o un inserimento di dati in una tabella \emph{DynamoDB}.
\\
Nel progetto è stato utilizzato per eseguire il codice che permette di
di effettuare la conversione delle immagini desiderate, recuperando una
configurazione da una tabella \emph{DynamoDB}, recuperando l'immagine da
convertire da un \emph{bucket S3} e salvando il risultato in un secondo
\emph{bucket S3}, dopo aver salvato le informazioni legate alla conversione in
una seconda tabella \emph{DynamoDB}.

\subsubsection{AWS DynamoDB}

\emph{AWS DynamoDB} è un database \emph{NoSQL} completamente gestito, che
supporta strutture di dati di tipo documento e di tipo chiave-valore. Questo
database supporta lettura e scrittura di dati senza degradamento delle
performance generali del servizio. \\
Nel progetto è stato utilizzato per salvare le informazioni legate alle
conversioni effettuate, come ad esempio il nome dell'immagine convertita, il
tipo di conversione effettuata e il momento in cui è stata effettuata la
conversione. È stato inoltre utilizzato per salvare le configurazioni di
conversione, come ad esempio il formato di output e la qualità dell'immagine
convertita.

\subsubsection{AWS S3}

\emph{AWS S3} è un servizio di \emph{storage} completamente gestito, che
permette di salvare e recuperare dati da una posizione di memoria detta
\emph{bucekt}. Supporta operazioni di lettura e scrittura di dati da parte di
altri servizi, ma può essere utilizzato per gestire \emph{backup} o per servire
siti \emph{web} statici. \\
Nel mio progetto è stato utilizzato per salvare le immagini da convertire e le
immagini convertite.

\subsubsection{AWS CloudWatch}

\emph{AWS CloudWatch} è un servizio di monitoraggio e gestione delle risorse
\emph{AWS} e delle applicazioni eseguite su \emph{AWS}. Il servizio permette di
raccogliere e monitorare dati di interesse, visualizzandoli in modo sintetico e
intuitivo. \\
Nel progetto è stato utilizzato per monitorare le conversioni effettuate e
raccogliere i \emph{log} presenti all'interno del codice eseguito dalla funzione
\emph{Lambda}.

\subsection{Versionamento}

\subsubsection{Git}

\emph{Git} è un sistema di controllo di versione distribuito, utilizzato per
tenere traccia delle modifiche effettuate al codice sorgente di un progetto
\glsfirstoccur\gls{software}.

\subsubsection{AWS CodeCommit}

\emph{AWS CodeCommit} è un servizio di controllo di versione completamente
gestito all'interno dell'ecosistema di \emph{AWS}, che permette di ospitare in modo sicuro e scalabile i \emph{repository}
\emph{Git} privati. \\

\subsection{Librerie e applicazioni esterne}

\subsubsection{Libvips}

\emph{Libvips} è una libreria \emph{open source} per la manipolazione di immagini
sviluppata in \emph{C}. La libreria è stata sviluppata per essere veloce e
utilizzare poca memoria, ed è ampiamente utilizzata da numerose applicazioni
\emph{web} e \emph{desktop}. \\
Nel progetto è stata utilizzata per effettuare la conversione delle immagini
di formato \emph{JPEG, WebP, PNG, SVG, TIFF e GIF} verso i formati \emph{JPEG} e
\emph{PNG}. La libreria permette di applicare all'immagine il profilo colore
desiderato, permette la rotazione automatica di un'immagine seguendo
l'orientamento presente dei \glsfirstoccur\gls{metadati}
\glsfirstoccur\gls{EXIF} e permette di scalare un'immagine seguendo la larghezza
e l'altezza desiderate. \cite{libvips}

\subsubsection{GhostScript}

\emph{GhostScript} è un interprete per il linguaggio di descrizione dei file
\emph{PostScript} ed \emph{EPS} (\emph{Encapsulated PostScript}).
Questo interprete permette la conversione delle immagini dai formati \emph{PS,
    EPS e AI} ai formati \emph{JPEG} e
\emph{PNG}. \\
Nel progetto viene richiamato direttamente dal codice \emph{Go} per effettuare
la conversione, specificando il formato di output desiderato. \cite{ghostscript}

\subsubsection{Darktable}

\emph{Darktable} è un'applicazione \emph{open source} per la gestione e la
modifica di immagini \emph{RAW}. Questa applicazione permette di convertire le
immagini \emph{RAW} in immagini nei formati \emph{JPEG} e \emph{PNG}.
L'applicazione presenta anche una interfaccia \glsfirstoccur\gls{CLI} che
permette di effettuare il processo di conversione immagini richiamandola
direttamente dal codice \emph{Go} del progetto.\\
L'applicazione permette di applicare un profilo colore desiderato e di impostare
la qualità dell'immagine convertita tramite un valore che varia da 0 a 100. \cite{darktable}

\subsubsection{Go-psd}

\emph{Go-psd} è una libreria \emph{open source} sviluppata in \emph{Go} che
permette di leggere e scrivere file \emph{PSD} (\emph{Photoshop Document}),
senza servirsi di \emph{software} esterni. \cite{go-psd}

\subsubsection{Imagemagick}

\emph{Imagemagick} è un \emph{software} \emph{open source} per la manipolazione
di immagini, che permette di convertire le immagini da un formato all'altro.
Nel progetto è stata utilizzata solo una minima parte del programma, ovvero
solo il comando \emph{identify}, che permette di recuperare i metadati dai file
di formato \emph{EPS, PS e AI}. \cite{imagemagick}

\subsection{Configurazione dell'ambiente di sviluppo del progetto}




